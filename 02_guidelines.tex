Here is a version of the guidelines on thesis preparation with minor edits. Please refer to the original document \citep{guidelines} for the most update and accurate information about thesis preparation, especially about the thesis submission protocol. 
\section{Introduction}
The guidelines in this document seek to ensure that theses are presented in a form suitable for library cataloging, preserving and access by users. The thesis will take its place in the library as a product of original thinking, research, and writing; its form must be comparable to other published works.

These guidelines cover the general rules of format and appearance. For content requirements, students should consult their Thesis Supervision Committee (TSC).

It is the student's responsibility to follow the requirements presented here. Thesis copies that do not meet these requirements will not be accepted.

Because of changes in requirements over time, students should not use existing library or departmental copies of theses as examples of current proper format.

\section{Originality}
\subsection{MPhil thesis}
An MPhil thesis shall consist of the student's own account of his/her investigations; be either a record of original work or an ordered, critical and thorough exposition of existing knowledge; be an integrated whole, presenting a coherent argument; give a critical assessment of the relevant literature, describe the method of research and its findings, and discuss those findings; and include a full bibliography.
\subsection{PhD thesis}
A PhD thesis shall consist of the student's own account of his/her investigations; make original, distinct contribution(s) to our knowledge of the subject and afford evidence or originality by the discovery of new facts and/or by the exercise of independent critical power; be an integrated whole with a coherent argument; give a critical assessment of the relevant literature, describe the method of research and its findings, and discuss those findings, particularly with regard to how these findings appear to the candidate to have advanced the study of the subject; include a full bibliography; and be of a standard to merit publication in whole, in part or in a revised form (for example, as a monograph or as a number of articles in learned journals).

\section{Components}
\subsection{Order}
A thesis should contain a Title page (containing thesis title, full name of the candidate, degree for which the thesis is submitted, name of the University, i.e. The Hong Kong University of Science and Technology, month and year of submission), Authorization page, Signature page, Acknowledgments, Table of contents, Lists of figures and tables, Abstract, Thesis body, Bibliography, Appendices and other addenda, if any.
\subsection{Authorization page}
On this page, students authorize the University to lend or reproduce the thesis. The copyright of the thesis as a literary work vests in its author (the student). The authorization gives HKUST Library a non-exclusive right to make it available for scholarly research.
\subsection{Signature page}
This page provides signatures of the thesis supervisor(s) and Department Head confirming that the thesis is satisfactory.
\subsection{Acknowledgments}
The student is required to declare, in this section, the extent to which assistance has been given by his/her faculty and staff, fellow students, external bodies or others in the collection of materials and data, the design and construction of apparatus, the performance of experiments, the analysis of data, and the preparation of the thesis (including editorial help). In addition, it is appropriate to recognize the supervision and advice given by the thesis supervisor(s) and members of TSC.
\subsection{Abstract}
Every copy of the thesis must have an English abstract, being a concise summary of the thesis, in 300 words or less.
\subsection{Bibliography}
The list of sources and references used should be presented in a standard format appropriate to the discipline; formatting should be consistent throughout.
\subsection{Sample pages}
Sample of both MPhil \citep{mphil} and PhD \citep{phd} theses are provided, with specific instructions for formatting page content (centering, spacing, etc.).

\section{Language, Style and Format}
\subsection{Language}
Theses should be written in English.

Students in the School of Humanities and Social Science who are pursuing research work in the areas of Chinese Studies, and who can demonstrate a need to use Chinese to write their theses should seek prior approval from the School via their thesis supervisor and the divisional head. If approval is granted, students are also required to produce a translation of the title page, authorization page, signature page, table of contents and the abstract in English.

\subsection{Pagination}
All pages, starting with the Title page should be numbered. All page numbers should be centered, at the bottom of each page.

Page numbers of materials preceding the body of the text should be in small Roman numerals. Page numbers of the text, beginning with the first page of the first chapter and continuing through the bibliography, including any pages with tables, maps, figures, photographs, etc., and any subsequent appendices, should be in Arabic numerals.\footnote{That means the Title page will be page i; the first page of the first chapter will be page 1.}

Start a new page after each chapter or section but not after a sub-section.

\subsection{Format}
A conventional font, size 12-point, 10 to 12 characters per inch must be used. One-and-a-half line spacing should be used throughout the thesis, except for abstracts, indented quotations or footnotes where single line spacing may be used.

All margins---top, bottom, sides---should be consistently 25mm (or no more than 30mm) in width. The same margin should be used throughout a thesis. Exceptionally, margins of a different size may be used when the nature of the thesis requires it.

\subsection{Footnotes}
Footnotes may be placed at the bottom of the page, at the end of each chapter or after the end of the thesis body. Like references, footnotes should be presented in a standard format appropriate to the discipline. Both the position and format of footnotes should be consistent throughout the thesis.

\subsection{Appendices}
The format of each appended item should be consistent with the nature of that item, whether text, diagram, figure, etc., and should follow the guidelines for that item as listed here.

\subsection{Figures, Tables and Illustrations}
Figures, tables, graphs, etc., should be positioned according to the scientific publication conventions of the discipline, e.g., interspersed in text or collected at the end of chapters. Charts, graphs, maps, and tables that are larger than a standard page should be provided as appendices.

\subsection{Photographs/Images}
High contrast photos should be used because they reproduce well. Photographs with a glossy finish and those with dark backgrounds should be avoided. Images should be dense enough to provide 300 ppi for printing and 72 dpi for viewing.

\subsection{Additional Materials}
Raw files, datasets, media files, and high resolution photographs/images of any format can be included. \footnote{Students should get approval from their Department Head before deviating from any of the above requirements concerning paper size, font, margins, etc.}

\section{Creating PDF files}
Theses must be submitted in PDF format. Providing a properly generated PDF file ensures the manuscript can be read using different platforms (Windows, Mac, etc.), that it displays as intended, and that it can be readily indexed.

Before submitting the PDF file, please use the the HKUST E-thesis PDF Preflight application to ensure all fonts should be embedded; image resolution should be dense enough to provide 300 ppi for printing and 72 dpi for viewing; the final thesis should be submitted as a single PDF file and PDF files should NOT be encrypted, as text cannot be extracted from encrypted PDFs for full text indexing or storage. Encrypted PDF files will NOT be accepted.

\section{Thesis Submission Protocol}
The final thesis must be free from typographical, grammatical and other errors when submitted to the Thesis Submission System. In particular, the thesis supervisor and the Department Head/program director should not sign off on the final thesis that is not, to the best of their knowledge, free of errors.

For examination purpose, sufficient hard or electronic thesis copies are to be submitted to the Department at least four weeks before the thesis examination. The number of copies required will depend on the number of examiners.

Students should submit the draft thesis to the iThenticate platform for originality check. The draft thesis together with the iThenticate report should be submitted to the Department no less than four weeks before the thesis examination.

On successful completion of the thesis examination, and after any required corrections, students must submit a copy of the final thesis (either hard/electronic) to their Department, which will arrange for the appropriate signatures of approval to be obtained.
For final theses which have been graded “Passed subject to minor corrections” or “Passed subject to major corrections”, students are required to submit the thesis for originality check via iThenticate. The iThenticate report should be handed in to the thesis supervisor(s), and the Thesis Examination Committee if applicable, for review and endorsement via their Department.
The Department will then return the signed Signature Page to the candidate.
The candidate will upload and submit the Signature Page and the Authorization Page as a PDF file and the final thesis as another PDF file to the University’s Thesis Submission System. The candidate does not need to replace the two unsigned pages in the thesis PDF with the scanned signature pages. The final thesis will be forwarded to the thesis supervisor(s) for approval via the Thesis Submission System.

\section{Copyright}
According to the University’s Intellectual Property Policy, students shall own the copyright in respect of their written coursework, theses, papers and publications themselves as a whole as literary works.

\section{Thesis Binding}
Students may wish to keep personal copies of their thesis. They may arrange for such copies on their own and at their own expense. Service from MTPC of the University is one option (details below). Students may explore other binderies for the binding service. In any case, the binding of the thesis must correspond with the University regulations.
